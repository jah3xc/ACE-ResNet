\documentclass[12pt]{article}
\usepackage[margin=1.0in]{geometry}
\usepackage{graphicx}
\usepackage{subcaption}
\usepackage{cleveref}
\usepackage{indentfirst}
\usepackage{amsmath}
\usepackage{listings}
\usepackage[title]{appendix}
\usepackage{siunitx} % Required for alignment
\usepackage{mdframed}

\begin{document}

	
\begin{center}
	\huge{Exploring the Effects of the Adaptive Cosine Estimator on Convolutional Neural Network Performance for Hyperspectral Imagery Classification} \\
	\vspace{5mm}
	\large{Alex Hurt}
\end{center}

%1) ABSTRACT. why care, what method did you use, what data did you explore, what did you find!
%
\section{Abstract}

Hyperspectral Imagery (HSI) Classification is a difficult but important task in Remote Sensing.
%
As hyperspectral sensors become more popular, the ability to perform classification at the hyperspectral level will grow more and more important.
%
HSI has many challenges that other image sources (i.e. RGB) do not have, such as labeling at the pixel level rather than image level and a so-called curse of dimensionality.
%
In recent years, several researchers have attempted to perform HSI classification with a variety of methods.
%
With the rise of deep learning, one of the most popular methods for HSI classification is Convolutional Neural Networks (CNN).
%
While HSI have several inherent challenges, so too do Convolutional Neural Networks, and understanding the limitations of CNN is critical to building a well performing classifier.
%
Recently, researchers have tried several tweaks to the standard CNN to improve HSI classification performance, such as using deformable convolutions by Zhu \textit{et al.} \cite{zhu_deformable_2018}. 
%
Another approach was taken by Paoletti \textit{et al.}, who used a CNN combined with capsule networks for HSI classification \cite{paoletti_capsule_2018}.

While numerous approaches have been taken for HSI classification, most techniques do not utilize the amount of spectral information contained in HSI when performing classification.
%
To that end, I will utilize a method of detection in HSI given sample object signatures, known as the Adaptive Cosine Estimator (ACE).
%
ACE allows detection and produces scoring at the pixel level within HSI given the signature of the objects of interest. 
%
In this project, I will combine the ACE technique with a CNN to produce a custom neural network architecture that involves preprocessing samples with ACE, named \textit{ACENet}.
%
ACENet combines the spectral classification methods of ACE with the spatial classification methods utilized by CNN to produce an architecture that can better classify HSI.
%
More details of this new architecture are described in Section \ref{sec:methods}.
%
Following the description of methods, I describe the data used for experimentation in Section \ref{sec:data} and then the design, results, and analysis of experiments in \ref{sec:experiments} before drawing final conclusions in Section \ref{sec:conclusion}.


%2) METHODS. describe your algorithm(s). your goal, show me you KNOW it
%
\section{Methods}\label{sec:methods}

%3) DATA. document your data
%
\section{Data}\label{sec:data}

%4) EXPERIMENTS and RESULTS. what did you run and what did you find. Yes, document, but I want to see your ANALYSIS!
%
\section{Experiments}\label{sec:experiments}

%5) CONCLUSIONS. summarize it folks!
\section{Conclusion}\label{sec:conclusion}


%%%%%%%%%%%%%%%%
%		 REFERENCES
%%%%%%%%%%%%%%%%%
\newpage
\bibliography{refs} 
\bibliographystyle{ieeetr}


\end{document}
